% Options for packages loaded elsewhere
% Options for packages loaded elsewhere
\PassOptionsToPackage{unicode}{hyperref}
\PassOptionsToPackage{hyphens}{url}
%
\documentclass[
  11pt,
  letterpaper,
]{book}
\usepackage{xcolor}
\usepackage[a4paper]{geometry}
\usepackage{amsmath,amssymb}
\setcounter{secnumdepth}{5}
\usepackage{iftex}
\ifPDFTeX
  \usepackage[T1]{fontenc}
  \usepackage[utf8]{inputenc}
  \usepackage{textcomp} % provide euro and other symbols
\else % if luatex or xetex
  \usepackage{unicode-math} % this also loads fontspec
  \defaultfontfeatures{Scale=MatchLowercase}
  \defaultfontfeatures[\rmfamily]{Ligatures=TeX,Scale=1}
\fi
\usepackage{lmodern}
\ifPDFTeX\else
  % xetex/luatex font selection
\fi
% Use upquote if available, for straight quotes in verbatim environments
\IfFileExists{upquote.sty}{\usepackage{upquote}}{}
\IfFileExists{microtype.sty}{% use microtype if available
  \usepackage[]{microtype}
  \UseMicrotypeSet[protrusion]{basicmath} % disable protrusion for tt fonts
}{}
\makeatletter
\@ifundefined{KOMAClassName}{% if non-KOMA class
  \IfFileExists{parskip.sty}{%
    \usepackage{parskip}
  }{% else
    \setlength{\parindent}{0pt}
    \setlength{\parskip}{6pt plus 2pt minus 1pt}}
}{% if KOMA class
  \KOMAoptions{parskip=half}}
\makeatother
% Make \paragraph and \subparagraph free-standing
\makeatletter
\ifx\paragraph\undefined\else
  \let\oldparagraph\paragraph
  \renewcommand{\paragraph}{
    \@ifstar
      \xxxParagraphStar
      \xxxParagraphNoStar
  }
  \newcommand{\xxxParagraphStar}[1]{\oldparagraph*{#1}\mbox{}}
  \newcommand{\xxxParagraphNoStar}[1]{\oldparagraph{#1}\mbox{}}
\fi
\ifx\subparagraph\undefined\else
  \let\oldsubparagraph\subparagraph
  \renewcommand{\subparagraph}{
    \@ifstar
      \xxxSubParagraphStar
      \xxxSubParagraphNoStar
  }
  \newcommand{\xxxSubParagraphStar}[1]{\oldsubparagraph*{#1}\mbox{}}
  \newcommand{\xxxSubParagraphNoStar}[1]{\oldsubparagraph{#1}\mbox{}}
\fi
\makeatother

\usepackage{color}
\usepackage{fancyvrb}
\newcommand{\VerbBar}{|}
\newcommand{\VERB}{\Verb[commandchars=\\\{\}]}
\DefineVerbatimEnvironment{Highlighting}{Verbatim}{commandchars=\\\{\}}
% Add ',fontsize=\small' for more characters per line
\usepackage{framed}
\definecolor{shadecolor}{RGB}{241,243,245}
\newenvironment{Shaded}{\begin{snugshade}}{\end{snugshade}}
\newcommand{\AlertTok}[1]{\textcolor[rgb]{0.68,0.00,0.00}{#1}}
\newcommand{\AnnotationTok}[1]{\textcolor[rgb]{0.37,0.37,0.37}{#1}}
\newcommand{\AttributeTok}[1]{\textcolor[rgb]{0.40,0.45,0.13}{#1}}
\newcommand{\BaseNTok}[1]{\textcolor[rgb]{0.68,0.00,0.00}{#1}}
\newcommand{\BuiltInTok}[1]{\textcolor[rgb]{0.00,0.23,0.31}{#1}}
\newcommand{\CharTok}[1]{\textcolor[rgb]{0.13,0.47,0.30}{#1}}
\newcommand{\CommentTok}[1]{\textcolor[rgb]{0.37,0.37,0.37}{#1}}
\newcommand{\CommentVarTok}[1]{\textcolor[rgb]{0.37,0.37,0.37}{\textit{#1}}}
\newcommand{\ConstantTok}[1]{\textcolor[rgb]{0.56,0.35,0.01}{#1}}
\newcommand{\ControlFlowTok}[1]{\textcolor[rgb]{0.00,0.23,0.31}{\textbf{#1}}}
\newcommand{\DataTypeTok}[1]{\textcolor[rgb]{0.68,0.00,0.00}{#1}}
\newcommand{\DecValTok}[1]{\textcolor[rgb]{0.68,0.00,0.00}{#1}}
\newcommand{\DocumentationTok}[1]{\textcolor[rgb]{0.37,0.37,0.37}{\textit{#1}}}
\newcommand{\ErrorTok}[1]{\textcolor[rgb]{0.68,0.00,0.00}{#1}}
\newcommand{\ExtensionTok}[1]{\textcolor[rgb]{0.00,0.23,0.31}{#1}}
\newcommand{\FloatTok}[1]{\textcolor[rgb]{0.68,0.00,0.00}{#1}}
\newcommand{\FunctionTok}[1]{\textcolor[rgb]{0.28,0.35,0.67}{#1}}
\newcommand{\ImportTok}[1]{\textcolor[rgb]{0.00,0.46,0.62}{#1}}
\newcommand{\InformationTok}[1]{\textcolor[rgb]{0.37,0.37,0.37}{#1}}
\newcommand{\KeywordTok}[1]{\textcolor[rgb]{0.00,0.23,0.31}{\textbf{#1}}}
\newcommand{\NormalTok}[1]{\textcolor[rgb]{0.00,0.23,0.31}{#1}}
\newcommand{\OperatorTok}[1]{\textcolor[rgb]{0.37,0.37,0.37}{#1}}
\newcommand{\OtherTok}[1]{\textcolor[rgb]{0.00,0.23,0.31}{#1}}
\newcommand{\PreprocessorTok}[1]{\textcolor[rgb]{0.68,0.00,0.00}{#1}}
\newcommand{\RegionMarkerTok}[1]{\textcolor[rgb]{0.00,0.23,0.31}{#1}}
\newcommand{\SpecialCharTok}[1]{\textcolor[rgb]{0.37,0.37,0.37}{#1}}
\newcommand{\SpecialStringTok}[1]{\textcolor[rgb]{0.13,0.47,0.30}{#1}}
\newcommand{\StringTok}[1]{\textcolor[rgb]{0.13,0.47,0.30}{#1}}
\newcommand{\VariableTok}[1]{\textcolor[rgb]{0.07,0.07,0.07}{#1}}
\newcommand{\VerbatimStringTok}[1]{\textcolor[rgb]{0.13,0.47,0.30}{#1}}
\newcommand{\WarningTok}[1]{\textcolor[rgb]{0.37,0.37,0.37}{\textit{#1}}}

\usepackage{longtable,booktabs,array}
\usepackage{calc} % for calculating minipage widths
% Correct order of tables after \paragraph or \subparagraph
\usepackage{etoolbox}
\makeatletter
\patchcmd\longtable{\par}{\if@noskipsec\mbox{}\fi\par}{}{}
\makeatother
% Allow footnotes in longtable head/foot
\IfFileExists{footnotehyper.sty}{\usepackage{footnotehyper}}{\usepackage{footnote}}
\makesavenoteenv{longtable}
\usepackage{graphicx}
\makeatletter
\newsavebox\pandoc@box
\newcommand*\pandocbounded[1]{% scales image to fit in text height/width
  \sbox\pandoc@box{#1}%
  \Gscale@div\@tempa{\textheight}{\dimexpr\ht\pandoc@box+\dp\pandoc@box\relax}%
  \Gscale@div\@tempb{\linewidth}{\wd\pandoc@box}%
  \ifdim\@tempb\p@<\@tempa\p@\let\@tempa\@tempb\fi% select the smaller of both
  \ifdim\@tempa\p@<\p@\scalebox{\@tempa}{\usebox\pandoc@box}%
  \else\usebox{\pandoc@box}%
  \fi%
}
% Set default figure placement to htbp
\def\fps@figure{htbp}
\makeatother





\setlength{\emergencystretch}{3em} % prevent overfull lines

\providecommand{\tightlist}{%
  \setlength{\itemsep}{0pt}\setlength{\parskip}{0pt}}



 


\makeatletter
\@ifpackageloaded{bookmark}{}{\usepackage{bookmark}}
\makeatother
\makeatletter
\@ifpackageloaded{caption}{}{\usepackage{caption}}
\AtBeginDocument{%
\ifdefined\contentsname
  \renewcommand*\contentsname{Table of contents}
\else
  \newcommand\contentsname{Table of contents}
\fi
\ifdefined\listfigurename
  \renewcommand*\listfigurename{List of Figures}
\else
  \newcommand\listfigurename{List of Figures}
\fi
\ifdefined\listtablename
  \renewcommand*\listtablename{List of Tables}
\else
  \newcommand\listtablename{List of Tables}
\fi
\ifdefined\figurename
  \renewcommand*\figurename{Figure}
\else
  \newcommand\figurename{Figure}
\fi
\ifdefined\tablename
  \renewcommand*\tablename{Table}
\else
  \newcommand\tablename{Table}
\fi
}
\@ifpackageloaded{float}{}{\usepackage{float}}
\floatstyle{ruled}
\@ifundefined{c@chapter}{\newfloat{codelisting}{h}{lop}}{\newfloat{codelisting}{h}{lop}[chapter]}
\floatname{codelisting}{Listing}
\newcommand*\listoflistings{\listof{codelisting}{List of Listings}}
\makeatother
\makeatletter
\makeatother
\makeatletter
\@ifpackageloaded{caption}{}{\usepackage{caption}}
\@ifpackageloaded{subcaption}{}{\usepackage{subcaption}}
\makeatother
\makeatletter
\definecolor{QuartoInternalColor14}{rgb}{0.80,0.20,0.20}
\definecolor{QuartoInternalColor11}{rgb}{0.00,0.00,1.00}
\definecolor{QuartoInternalColor6}{rgb}{0.38,0.38,0.38}
\definecolor{QuartoInternalColor2}{rgb}{0,0,0}
\definecolor{QuartoInternalColor10}{rgb}{0.00,0.40,0.00}
\definecolor{QuartoInternalColor3}{rgb}{0.00,0.64,0.31}
\definecolor{QuartoInternalColor1}{rgb}{0.91,0.36,0.35}
\definecolor{QuartoInternalColor5}{rgb}{0.20,0.40,0.40}
\definecolor{QuartoInternalColor16}{rgb}{0.13,0.56,0.98}
\definecolor{QuartoInternalColor4}{rgb}{0.00,0.45,0.15}
\definecolor{QuartoInternalColor8}{rgb}{0.15,0.56,0.56}
\definecolor{QuartoInternalColor17}{rgb}{0.77,0.76,0.71}
\definecolor{QuartoInternalColor7}{rgb}{0.60,0.00,0.00}
\definecolor{QuartoInternalColor13}{rgb}{0.60,0.20,0.40}
\definecolor{QuartoInternalColor12}{rgb}{0.74,0.74,0.74}
\definecolor{QuartoInternalColor9}{rgb}{0.87,0.71,0.17}
\definecolor{QuartoInternalColor15}{rgb}{0.38,0.78,0.78}
\makeatother
\usepackage{bookmark}
\IfFileExists{xurl.sty}{\usepackage{xurl}}{} % add URL line breaks if available
\urlstyle{same}
\hypersetup{
  pdftitle={arcpy},
  pdfauthor={Lakshmi Kantakumar},
  hidelinks,
  pdfcreator={LaTeX via pandoc}}


\title{arcpy}
\author{Lakshmi Kantakumar}
\date{2025-09-08}
\begin{document}
\frontmatter
\maketitle

\renewcommand*\contentsname{Table of contents}
{
\setcounter{tocdepth}{1}
\tableofcontents
}

\mainmatter
\bookmarksetup{startatroot}

\chapter*{Preface}\label{preface}
\addcontentsline{toc}{chapter}{Preface}

\markboth{Preface}{Preface}

This is a Quarto book.

To learn more about Quarto books visit
\url{https://quarto.org/docs/books}.

\part{Part 1: Python Foundations for GIS}

\chapter{Introduction to Python for
GIS}\label{introduction-to-python-for-gis}

Python is one of the most widely used programming languages in the
world, and it plays a vital role in Geographic Information Systems
(GIS). In ArcGIS, Python is the primary scripting language used to
automate workflows, perform geospatial analysis, and extend ArcGIS
functionality.

This chapter introduces Python in the context of GIS and provides the
foundation you need to start writing Python scripts in ArcGIS.

\begin{center}\rule{0.5\linewidth}{0.5pt}\end{center}

\section{Why Python for ArcGIS?}\label{why-python-for-arcgis}

Python has become the scripting language of choice for ArcGIS because:

\begin{itemize}
\tightlist
\item
  It is \textbf{easy to learn} and beginner-friendly.\\
\item
  It allows \textbf{automation} of repetitive tasks.\\
\item
  It provides access to \textbf{ArcPy}, a powerful library for
  geoprocessing.\\
\item
  It integrates well with other Python libraries (\texttt{pandas},
  \texttt{numpy}, \texttt{matplotlib}).
\end{itemize}

ArcGIS Pro comes with its own Python environment, which makes it easy to
get started.

\begin{center}\rule{0.5\linewidth}{0.5pt}\end{center}

\section{Accessing Python in ArcGIS}\label{accessing-python-in-arcgis}

There are several ways to use Python with ArcGIS:

\begin{enumerate}
\def\labelenumi{\arabic{enumi}.}
\tightlist
\item
  \textbf{Python Window in ArcGIS Pro}

  \begin{itemize}
  \tightlist
  \item
    A built-in console for running quick Python commands.
  \end{itemize}
\item
  \textbf{Standalone Python Scripts}

  \begin{itemize}
  \tightlist
  \item
    Create \texttt{.py} files and run them inside ArcGIS Pro's
    environment.
  \end{itemize}
\item
  \textbf{Jupyter Notebooks}

  \begin{itemize}
  \tightlist
  \item
    Interactive way of combining code, results, and explanations (what
    we use here).
  \end{itemize}
\end{enumerate}

\begin{center}\rule{0.5\linewidth}{0.5pt}\end{center}

\section{Writing Your First Python Script in
ArcGIS}\label{writing-your-first-python-script-in-arcgis}

Let's begin with a simple Python program to print a message.

\begin{Shaded}
\begin{Highlighting}[]
\BuiltInTok{print}\NormalTok{(}\StringTok{"Hello, GIS World!"}\NormalTok{)}
\end{Highlighting}
\end{Shaded}

\begin{verbatim}
Hello, GIS World!
\end{verbatim}

\begin{center}\rule{0.5\linewidth}{0.5pt}\end{center}

\section{Using Python as a
Calculator}\label{using-python-as-a-calculator}

Python can also be used like a calculator:

\begin{Shaded}
\begin{Highlighting}[]
\NormalTok{a }\OperatorTok{=} \DecValTok{10}
\NormalTok{b }\OperatorTok{=} \DecValTok{5}
\NormalTok{sum\_ab }\OperatorTok{=}\NormalTok{ a }\OperatorTok{+}\NormalTok{ b}
\NormalTok{product\_ab }\OperatorTok{=}\NormalTok{ a }\OperatorTok{*}\NormalTok{ b}

\BuiltInTok{print}\NormalTok{(}\StringTok{"Sum:"}\NormalTok{, sum\_ab)}
\BuiltInTok{print}\NormalTok{(}\StringTok{"Product:"}\NormalTok{, product\_ab)}
\end{Highlighting}
\end{Shaded}

\begin{verbatim}
Sum: 15
Product: 50
\end{verbatim}

\begin{center}\rule{0.5\linewidth}{0.5pt}\end{center}

\section{Hello ArcPy}\label{hello-arcpy}

To connect Python with ArcGIS, we import the \textbf{ArcPy} module.

\begin{quote}
Note: This code will only work inside the ArcGIS Pro Python environment.
\end{quote}

\begin{Shaded}
\begin{Highlighting}[]
\ImportTok{import}\NormalTok{ arcpy}

\CommentTok{\# Check ArcPy version}
\BuiltInTok{print}\NormalTok{(}\StringTok{"ArcPy is ready! Version:"}\NormalTok{, arcpy.GetInstallInfo()[}\StringTok{"Version"}\NormalTok{])}
\end{Highlighting}
\end{Shaded}

\begin{verbatim}
ArcPy is ready! Version: 3.4
\end{verbatim}

\begin{center}\rule{0.5\linewidth}{0.5pt}\end{center}

\section{Example: List Feature Classes in a
Folder}\label{example-list-feature-classes-in-a-folder}

Let's try a small GIS-related script using ArcPy.

\begin{Shaded}
\begin{Highlighting}[]
\ImportTok{import}\NormalTok{ arcpy}

\CommentTok{\# Set workspace (change path to your local GIS data folder)}
\NormalTok{arcpy.env.workspace }\OperatorTok{=} \VerbatimStringTok{r"D:}\DecValTok{\textbackslash{}G}\VerbatimStringTok{IS\_Data}\ErrorTok{\textbackslash{}}\VerbatimStringTok{OSM}\DecValTok{\textbackslash{}s}\VerbatimStringTok{hape"}

\CommentTok{\# List all feature classes}
\NormalTok{feature\_classes }\OperatorTok{=}\NormalTok{ arcpy.ListFeatureClasses()}

\BuiltInTok{print}\NormalTok{(}\StringTok{"Feature Classes in workspace:"}\NormalTok{)}
\ControlFlowTok{for}\NormalTok{ fc }\KeywordTok{in}\NormalTok{ feature\_classes:}
    \BuiltInTok{print}\NormalTok{(}\StringTok{"{-}"}\NormalTok{, fc)}
\end{Highlighting}
\end{Shaded}

\begin{verbatim}
Feature Classes in workspace:
- buildings.shp
- landuse.shp
- natural.shp
- places.shp
- points.shp
- railways.shp
- roads.shp
- roads_utm44n.shp
- waterways.shp
\end{verbatim}

\begin{center}\rule{0.5\linewidth}{0.5pt}\end{center}

\section{Visual Guide: Ways to Use Python in
ArcGIS}\label{visual-guide-ways-to-use-python-in-arcgis}

\begin{Shaded}
\begin{Highlighting}[]
\NormalTok{flowchart TD}
\NormalTok{    A["Python Window (ArcGIS Pro)"] {-}{-}\textgreater{} D["Use Python for quick tasks"]}
\NormalTok{    B["Standalone Scripts (.py)"] {-}{-}\textgreater{} D}
\NormalTok{    C["Notebooks (Jupyter / Quarto)"] {-}{-}\textgreater{} D}
\NormalTok{    D["Automate \& Analyze GIS Data with ArcPy"]}
\end{Highlighting}
\end{Shaded}

\includegraphics[width=9.04in,height=2.31in]{chapters/part_1/chapter1_intro_python_gis_files/figure-latex/mermaid-figure-1.png}

\begin{center}\rule{0.5\linewidth}{0.5pt}\end{center}

\section{Summary}\label{summary}

In this chapter, we learned:\\
- Why Python is important for ArcGIS.\\
- Different ways to use Python in ArcGIS.\\
- How to write a simple Python script.\\
- How to connect to ArcPy and list GIS datasets.

\begin{center}\rule{0.5\linewidth}{0.5pt}\end{center}

\section{Exercise}\label{exercise}

Try the following tasks on your own:

\begin{enumerate}
\def\labelenumi{\arabic{enumi}.}
\tightlist
\item
  Write a Python script that multiplies two numbers and prints the
  result.\\
\item
  Modify the ArcPy script above to set your own folder path and list
  feature classes.\\
\item
  Run the \texttt{arcpy.GetInstallInfo()} command and note down your
  ArcGIS version.
\end{enumerate}

\begin{center}\rule{0.5\linewidth}{0.5pt}\end{center}

\chapter{Python Basics}\label{python-basics}

In this chapter, we will review the basic Python concepts needed for GIS
scripting.\\
Understanding these fundamentals will help you write more effective
scripts in ArcGIS.

\begin{center}\rule{0.5\linewidth}{0.5pt}\end{center}

\section{Variables and Data Types}\label{variables-and-data-types}

A \textbf{variable} is a named storage that holds data in memory, which
can be used and modified later.\\
In Python, variables are created when you assign a value using the
\texttt{=} operator.

\subsection{Rules for variable names}\label{rules-for-variable-names}

\begin{itemize}
\tightlist
\item
  Must start with a letter or underscore (\texttt{\_})\\
\item
  Cannot start with a number\\
\item
  Can only contain letters, numbers, and underscores\\
\item
  Case-sensitive (\texttt{Name}, \texttt{name}, and \texttt{NAME} are
  different variables)
\end{itemize}

\begin{center}\rule{0.5\linewidth}{0.5pt}\end{center}

\subsection{Common Python Data Types}\label{common-python-data-types}

\begin{itemize}
\tightlist
\item
  \textbf{int (Integer)} → whole numbers (e.g., \texttt{10},
  \texttt{-5}, \texttt{2025})\\
\item
  \textbf{float (Floating Point)} → decimal numbers (e.g.,
  \texttt{3.14}, \texttt{-0.5})\\
\item
  \textbf{str (String)} → text data (e.g., \texttt{"Hello\ GIS"},
  \texttt{\textquotesingle{}ArcPy\textquotesingle{}})\\
\item
  \textbf{list} → ordered collection of items (e.g.,
  \texttt{{[}1,\ 2,\ 3,\ "ArcGIS"{]}})
\item
  \textbf{dict (Dictionary)} → key-value pairs (e.g.,
  \texttt{\{"name":\ "India",\ "population":\ 1400\}})
\item
  \textbf{bool (Boolean)} → True/False values (e.g., \texttt{True},
  \texttt{False})
\end{itemize}

\begin{center}\rule{0.5\linewidth}{0.5pt}\end{center}

\subsection{Example: Declaring
Variables}\label{example-declaring-variables}

\begin{Shaded}
\begin{Highlighting}[]
\CommentTok{\# Integer}
\NormalTok{integer\_num }\OperatorTok{=} \DecValTok{10}

\CommentTok{\# Float}
\NormalTok{decimal\_num }\OperatorTok{=} \FloatTok{3.14}

\CommentTok{\# String}
\NormalTok{text\_value }\OperatorTok{=} \StringTok{"Hello GIS"}

\CommentTok{\# Boolean}
\NormalTok{is\_active }\OperatorTok{=} \VariableTok{True}

\CommentTok{\# List}
\NormalTok{my\_list }\OperatorTok{=}\NormalTok{ [}\DecValTok{1}\NormalTok{, }\DecValTok{2}\NormalTok{, }\DecValTok{3}\NormalTok{, }\StringTok{"ArcGIS"}\NormalTok{]}

\CommentTok{\# Dictionary}
\NormalTok{my\_dict }\OperatorTok{=}\NormalTok{ \{}\StringTok{"name"}\NormalTok{: }\StringTok{"India"}\NormalTok{, }\StringTok{"population"}\NormalTok{: }\DecValTok{1400}\NormalTok{\}}

\BuiltInTok{print}\NormalTok{(integer\_num, }\BuiltInTok{type}\NormalTok{(integer\_num))}
\BuiltInTok{print}\NormalTok{(decimal\_num, }\BuiltInTok{type}\NormalTok{(decimal\_num))}
\BuiltInTok{print}\NormalTok{(text\_value, }\BuiltInTok{type}\NormalTok{(text\_value))}
\BuiltInTok{print}\NormalTok{(is\_active, }\BuiltInTok{type}\NormalTok{(is\_active))}
\BuiltInTok{print}\NormalTok{(my\_list, }\BuiltInTok{type}\NormalTok{(my\_list))}
\BuiltInTok{print}\NormalTok{(my\_dict, }\BuiltInTok{type}\NormalTok{(my\_dict))}
\end{Highlighting}
\end{Shaded}

\begin{verbatim}
10 <class 'int'>
3.14 <class 'float'>
Hello GIS <class 'str'>
True <class 'bool'>
[1, 2, 3, 'ArcGIS'] <class 'list'>
{'name': 'India', 'population': 1400} <class 'dict'>
\end{verbatim}

\begin{center}\rule{0.5\linewidth}{0.5pt}\end{center}

\section{Operators and Expressions}\label{operators-and-expressions}

Operators allow us to perform operations on variables and values.

\begin{itemize}
\tightlist
\item
  Arithmetic Operators: \texttt{+}, \texttt{-}, \texttt{*}, \texttt{/},
  \texttt{//}, \texttt{\%}, \texttt{**}\\
\item
  Comparison Operators: \texttt{==}, \texttt{!=}, \texttt{\textless{}},
  \texttt{\textgreater{}}, \texttt{\textless{}=},
  \texttt{\textgreater{}=}\\
\item
  Logical Operators: \texttt{and}, \texttt{or}, \texttt{not}
\end{itemize}

\begin{Shaded}
\begin{Highlighting}[]
\NormalTok{a }\OperatorTok{=} \DecValTok{15}
\NormalTok{b }\OperatorTok{=} \DecValTok{4}

\BuiltInTok{print}\NormalTok{(}\StringTok{"Addition:"}\NormalTok{, a }\OperatorTok{+}\NormalTok{ b)}
\BuiltInTok{print}\NormalTok{(}\StringTok{"Division:"}\NormalTok{, a }\OperatorTok{/}\NormalTok{ b)}
\BuiltInTok{print}\NormalTok{(}\StringTok{"Floor Division:"}\NormalTok{, a }\OperatorTok{//}\NormalTok{ b)}
\BuiltInTok{print}\NormalTok{(}\StringTok{"Exponent:"}\NormalTok{, a }\OperatorTok{**}\NormalTok{ b)}

\BuiltInTok{print}\NormalTok{(}\StringTok{"Equal?"}\NormalTok{, a }\OperatorTok{==}\NormalTok{ b)}
\BuiltInTok{print}\NormalTok{(}\StringTok{"Greater?"}\NormalTok{, a }\OperatorTok{\textgreater{}}\NormalTok{ b)}
\BuiltInTok{print}\NormalTok{(}\StringTok{"Logical AND:"}\NormalTok{, a }\OperatorTok{\textgreater{}} \DecValTok{10} \KeywordTok{and}\NormalTok{ b }\OperatorTok{\textless{}} \DecValTok{10}\NormalTok{)}
\end{Highlighting}
\end{Shaded}

\begin{verbatim}
Addition: 19
Division: 3.75
Floor Division: 3
Exponent: 50625
Equal? False
Greater? True
Logical AND: True
\end{verbatim}

\begin{center}\rule{0.5\linewidth}{0.5pt}\end{center}

\section{Input/Output and Simple
Programs}\label{inputoutput-and-simple-programs}

In interactive Python (like IDLE or terminal), you can take input with
\texttt{input()}.

\begin{Shaded}
\begin{Highlighting}[]
\CommentTok{\# Example: calculate area of a rectangle}
\CommentTok{\# (In a real script, use: length = float(input("Enter length: ")))}

\NormalTok{length }\OperatorTok{=} \FloatTok{10.0}   \CommentTok{\# pretend the user typed 10}
\NormalTok{width }\OperatorTok{=} \FloatTok{5.0}     \CommentTok{\# pretend the user typed 5}

\NormalTok{area }\OperatorTok{=}\NormalTok{ length }\OperatorTok{*}\NormalTok{ width}
\BuiltInTok{print}\NormalTok{(}\StringTok{"The area of rectangle is:"}\NormalTok{, area)}
\end{Highlighting}
\end{Shaded}

\begin{verbatim}
The area of rectangle is: 50.0
\end{verbatim}

\begin{center}\rule{0.5\linewidth}{0.5pt}\end{center}

\section{Working with Strings}\label{working-with-strings}

Strings are important in GIS because we often work with file paths,
names, and attribute data.

\begin{Shaded}
\begin{Highlighting}[]
\CommentTok{\# String operations}
\NormalTok{path }\OperatorTok{=} \StringTok{"D:/GIS\_Data/OSM/shape/roads.shp"}

\BuiltInTok{print}\NormalTok{(}\StringTok{"Length of string:"}\NormalTok{, }\BuiltInTok{len}\NormalTok{(path))}
\BuiltInTok{print}\NormalTok{(}\StringTok{"Uppercase:"}\NormalTok{, path.upper())}
\BuiltInTok{print}\NormalTok{(}\StringTok{"Does it end with .shp?"}\NormalTok{, path.endswith(}\StringTok{".shp"}\NormalTok{))}
\BuiltInTok{print}\NormalTok{(}\StringTok{"Replace extension:"}\NormalTok{, path.replace(}\StringTok{".shp"}\NormalTok{, }\StringTok{".geojson"}\NormalTok{))}

\CommentTok{\# Splitting file path}
\NormalTok{parts }\OperatorTok{=}\NormalTok{ path.split(}\StringTok{"}\CharTok{\textbackslash{}\textbackslash{}}\StringTok{"}\NormalTok{)}
\BuiltInTok{print}\NormalTok{(}\StringTok{"Folder structure:"}\NormalTok{, parts)}
\end{Highlighting}
\end{Shaded}

\begin{verbatim}
Length of string: 31
Uppercase: D:/GIS_DATA/OSM/SHAPE/ROADS.SHP
Does it end with .shp? True
Replace extension: D:/GIS_Data/OSM/shape/roads.geojson
Folder structure: ['D:/GIS_Data/OSM/shape/roads.shp']
\end{verbatim}

\begin{center}\rule{0.5\linewidth}{0.5pt}\end{center}

\section{Summary}\label{summary-1}

In this chapter, we reviewed:\\
- Variables and common data types in Python.\\
- Basic operators and expressions.\\
- Simple input/output programs.\\
- String operations useful for GIS tasks.

\begin{center}\rule{0.5\linewidth}{0.5pt}\end{center}

\section{Exercise}\label{exercise-1}

Try the following tasks:

\begin{enumerate}
\def\labelenumi{\arabic{enumi}.}
\tightlist
\item
  Create a dictionary with details of your city (name, population,
  state).\\
\item
  Write a program that asks the user for two numbers and prints their
  sum and product.\\
\item
  Modify a file path string to change the extension from \texttt{.tif}
  to \texttt{.jpg}.
\end{enumerate}

\begin{center}\rule{0.5\linewidth}{0.5pt}\end{center}

\chapter{Control Structures in
Python}\label{control-structures-in-python}

Control structures are used to control the flow of a program.\\
In this chapter, we will learn about conditional statements and loops,
which are essential for automating GIS tasks.

\begin{center}\rule{0.5\linewidth}{0.5pt}\end{center}

\section{If-Else Conditions}\label{if-else-conditions}

The \textbf{if-else} statement is used to make decisions in a program.

\begin{Shaded}
\begin{Highlighting}[]
\NormalTok{x }\OperatorTok{=} \DecValTok{20}

\ControlFlowTok{if}\NormalTok{ x }\OperatorTok{\textgreater{}} \DecValTok{10}\NormalTok{:}
    \BuiltInTok{print}\NormalTok{(}\StringTok{"x is greater than 10"}\NormalTok{)}
\ControlFlowTok{else}\NormalTok{:}
    \BuiltInTok{print}\NormalTok{(}\StringTok{"x is 10 or less"}\NormalTok{)}
\end{Highlighting}
\end{Shaded}

\begin{verbatim}
x is greater than 10
\end{verbatim}

We can also use \textbf{elif} for multiple conditions.

\begin{Shaded}
\begin{Highlighting}[]
\NormalTok{value }\OperatorTok{=} \DecValTok{0}

\ControlFlowTok{if}\NormalTok{ value }\OperatorTok{\textgreater{}} \DecValTok{0}\NormalTok{:}
    \BuiltInTok{print}\NormalTok{(}\StringTok{"Positive number"}\NormalTok{)}
\ControlFlowTok{elif}\NormalTok{ value }\OperatorTok{\textless{}} \DecValTok{0}\NormalTok{:}
    \BuiltInTok{print}\NormalTok{(}\StringTok{"Negative number"}\NormalTok{)}
\ControlFlowTok{else}\NormalTok{:}
    \BuiltInTok{print}\NormalTok{(}\StringTok{"Zero"}\NormalTok{)}
\end{Highlighting}
\end{Shaded}

\begin{verbatim}
Zero
\end{verbatim}

\begin{center}\rule{0.5\linewidth}{0.5pt}\end{center}

\section{Loops: For and While}\label{loops-for-and-while}

Loops help us repeat a block of code multiple times.

\subsection{For Loop}\label{for-loop}

A \textbf{for loop} iterates over a sequence (list, range, string, etc.)
and executes code for each item.

\begin{Shaded}
\begin{Highlighting}[]
\CommentTok{\# Print numbers 1 to 5}
\ControlFlowTok{for}\NormalTok{ i }\KeywordTok{in} \BuiltInTok{range}\NormalTok{(}\DecValTok{1}\NormalTok{, }\DecValTok{6}\NormalTok{):}
    \BuiltInTok{print}\NormalTok{(}\StringTok{"Number:"}\NormalTok{, i)}
\end{Highlighting}
\end{Shaded}

\begin{verbatim}
Number: 1
Number: 2
Number: 3
Number: 4
Number: 5
\end{verbatim}

\subsection{While Loop}\label{while-loop}

A \textbf{while loop} executes code as long as a given condition is
True.

\begin{Shaded}
\begin{Highlighting}[]
\CommentTok{\# Print numbers until condition is met}
\NormalTok{n }\OperatorTok{=} \DecValTok{1}
\ControlFlowTok{while}\NormalTok{ n }\OperatorTok{\textless{}=} \DecValTok{5}\NormalTok{:}
    \BuiltInTok{print}\NormalTok{(}\StringTok{"n ="}\NormalTok{, n)}
\NormalTok{    n }\OperatorTok{+=} \DecValTok{1}
\end{Highlighting}
\end{Shaded}

\begin{verbatim}
n = 1
n = 2
n = 3
n = 4
n = 5
\end{verbatim}

\begin{center}\rule{0.5\linewidth}{0.5pt}\end{center}

\section{Iterating Over Lists and
Dictionaries}\label{iterating-over-lists-and-dictionaries}

We often need to loop through collections of data.

\subsection{Looping Over a List}\label{looping-over-a-list}

\begin{Shaded}
\begin{Highlighting}[]
\NormalTok{layers }\OperatorTok{=}\NormalTok{ [}\StringTok{"roads.shp"}\NormalTok{, }\StringTok{"rivers.shp"}\NormalTok{, }\StringTok{"villages.shp"}\NormalTok{]}

\ControlFlowTok{for}\NormalTok{ layer }\KeywordTok{in}\NormalTok{ layers:}
    \BuiltInTok{print}\NormalTok{(}\StringTok{"Processing:"}\NormalTok{, layer)}
\end{Highlighting}
\end{Shaded}

\begin{verbatim}
Processing: roads.shp
Processing: rivers.shp
Processing: villages.shp
\end{verbatim}

\subsection{Looping Over a Dictionary}\label{looping-over-a-dictionary}

\begin{Shaded}
\begin{Highlighting}[]
\NormalTok{city }\OperatorTok{=}\NormalTok{ \{}\StringTok{"name"}\NormalTok{: }\StringTok{"Delhi"}\NormalTok{, }\StringTok{"population"}\NormalTok{: }\DecValTok{19000000}\NormalTok{, }\StringTok{"country"}\NormalTok{: }\StringTok{"India"}\NormalTok{\}}

\ControlFlowTok{for}\NormalTok{ key, value }\KeywordTok{in}\NormalTok{ city.items():}
    \BuiltInTok{print}\NormalTok{(key, }\StringTok{":"}\NormalTok{, value)}
\end{Highlighting}
\end{Shaded}

\begin{verbatim}
name : Delhi
population : 19000000
country : India
\end{verbatim}

\begin{center}\rule{0.5\linewidth}{0.5pt}\end{center}

\section{Practical Example: Looping Over Shapefile Names in a
Folder}\label{practical-example-looping-over-shapefile-names-in-a-folder}

In GIS automation, we often need to process multiple shapefiles.\\
This example demonstrates looping through shapefiles using ArcPy.

\begin{Shaded}
\begin{Highlighting}[]
\ImportTok{import}\NormalTok{ arcpy}

\CommentTok{\# Set workspace folder containing shapefiles}
\NormalTok{arcpy.env.workspace }\OperatorTok{=} \VerbatimStringTok{r"D:/GIS\_Data/OSM/shape"}

\CommentTok{\# List all shapefiles in the folder}
\NormalTok{shapefiles }\OperatorTok{=}\NormalTok{ arcpy.ListFeatureClasses()}

\ControlFlowTok{for}\NormalTok{ shp }\KeywordTok{in}\NormalTok{ shapefiles:}
    \BuiltInTok{print}\NormalTok{(}\StringTok{"Shapefile found:"}\NormalTok{, shp)}
    \CommentTok{\# Example: describe each shapefile}
\NormalTok{    desc }\OperatorTok{=}\NormalTok{ arcpy.Describe(shp)}
    \BuiltInTok{print}\NormalTok{(}\StringTok{" {-} Shape Type:"}\NormalTok{, desc.shapeType)}
    \BuiltInTok{print}\NormalTok{(}\StringTok{" {-} Feature Count:"}\NormalTok{, arcpy.GetCount\_management(shp))}
\end{Highlighting}
\end{Shaded}

\begin{verbatim}
Shapefile found: buildings.shp
 - Shape Type: Polygon
 - Feature Count: 19884
Shapefile found: landuse.shp
 - Shape Type: Polygon
 - Feature Count: 43
Shapefile found: natural.shp
 - Shape Type: Polygon
 - Feature Count: 16
Shapefile found: places.shp
 - Shape Type: Point
 - Feature Count: 47
Shapefile found: points.shp
 - Shape Type: Point
 - Feature Count: 344
Shapefile found: railways.shp
 - Shape Type: Polyline
 - Feature Count: 84
Shapefile found: roads.shp
 - Shape Type: Polyline
 - Feature Count: 2826
Shapefile found: roads_utm44n.shp
 - Shape Type: Polyline
 - Feature Count: 2826
Shapefile found: waterways.shp
 - Shape Type: Polyline
 - Feature Count: 18
\end{verbatim}

\begin{center}\rule{0.5\linewidth}{0.5pt}\end{center}

\section{Summary}\label{summary-2}

In this chapter, we learned:\\
- How to use \textbf{if-else conditions}.\\
- How to write \textbf{for} and \textbf{while} loops.\\
- How to iterate through \textbf{lists} and \textbf{dictionaries}.\\
- A practical example of looping through shapefiles in a folder.

\begin{center}\rule{0.5\linewidth}{0.5pt}\end{center}

\section{Exercise}\label{exercise-2}

Try the following tasks:

\begin{enumerate}
\def\labelenumi{\arabic{enumi}.}
\tightlist
\item
  Write a program that checks whether a number entered by the user is
  even or odd.\\
\item
  Create a list of five file names and use a loop to print each one.\\
\item
  Modify the ArcPy script to print only polygon shapefiles from a
  folder.
\end{enumerate}

\begin{center}\rule{0.5\linewidth}{0.5pt}\end{center}

\chapter{Functions and Modules in
Python}\label{functions-and-modules-in-python}

Functions and modules allow us to organize and reuse code effectively.\\
Instead of writing the same code again and again, we can wrap logic
inside a \textbf{function} or use a \textbf{module} (a collection of
functions).

In GIS scripting, functions help structure workflows (e.g., buffer →
clip → calculate area), while modules provide useful tools for different
tasks (e.g., file management with \texttt{os}, GIS tools with
\texttt{arcpy}).

\begin{center}\rule{0.5\linewidth}{0.5pt}\end{center}

\section{Defining Functions}\label{defining-functions}

A \textbf{function} is a block of code that runs only when it is
called.\\
Functions can:

\begin{itemize}
\tightlist
\item
  Take inputs (called \textbf{parameters})\\
\item
  Perform some operations\\
\item
  Optionally return an output using the \texttt{return} statement
\end{itemize}

To define a function in Python, we use the \textbf{\texttt{def} keyword}
followed by:\\
1. The function name\\
2. Parentheses \texttt{()} (with parameters inside, if any)\\
3. A colon \texttt{:}\\
4. An indented block of code (the function body)

\textbf{General syntax:}

\begin{Shaded}
\begin{Highlighting}[]
\KeywordTok{def}\NormalTok{ function\_name(parameters):}
    \CommentTok{\# code block}
    \ControlFlowTok{return}\NormalTok{ result   }\CommentTok{\# optional}
\end{Highlighting}
\end{Shaded}

\section{Defining Functions}\label{defining-functions-1}

A \textbf{function} is a block of code that runs only when called.
Functions can take inputs (parameters) and return outputs.

\subsection{Example: A simple function}\label{example-a-simple-function}

\begin{Shaded}
\begin{Highlighting}[]
\CommentTok{\# Simple function without parameters}
\KeywordTok{def}\NormalTok{ say\_hello():}
    \BuiltInTok{print}\NormalTok{(}\StringTok{"Hello from Python!"}\NormalTok{)}

\NormalTok{say\_hello()}
\end{Highlighting}
\end{Shaded}

\begin{verbatim}
Hello from Python!
\end{verbatim}

\subsection{Example: Function with parameters and return
value}\label{example-function-with-parameters-and-return-value}

\begin{Shaded}
\begin{Highlighting}[]
\KeywordTok{def}\NormalTok{ add\_numbers(a, b):}
    \CommentTok{"""This function takes two numbers and returns their sum."""}
\NormalTok{    result }\OperatorTok{=}\NormalTok{ a }\OperatorTok{+}\NormalTok{ b}
    \ControlFlowTok{return}\NormalTok{ result}

\CommentTok{\# Call the function with arguments}
\BuiltInTok{print}\NormalTok{(}\StringTok{"Sum of 5 and 7 is:"}\NormalTok{, add\_numbers(}\DecValTok{5}\NormalTok{, }\DecValTok{7}\NormalTok{))}
\end{Highlighting}
\end{Shaded}

\begin{verbatim}
Sum of 5 and 7 is: 12
\end{verbatim}

\subsection{Example: Function with default
parameters}\label{example-function-with-default-parameters}

\begin{Shaded}
\begin{Highlighting}[]
\KeywordTok{def}\NormalTok{ greet(name}\OperatorTok{=}\StringTok{"GIS Learner"}\NormalTok{):}
    \CommentTok{"""Greets the person with a default value if no name is provided."""}
    \BuiltInTok{print}\NormalTok{(}\StringTok{"Welcome,"}\NormalTok{, name)}

\NormalTok{greet(}\StringTok{"Lakshmi"}\NormalTok{)    }\CommentTok{\# custom value}
\NormalTok{greet()             }\CommentTok{\# uses default value}
\end{Highlighting}
\end{Shaded}

\begin{verbatim}
Welcome, Lakshmi
Welcome, GIS Learner
\end{verbatim}

\begin{center}\rule{0.5\linewidth}{0.5pt}\end{center}

\section{Using Modules in Python}\label{using-modules-in-python}

A \textbf{module} is a file that contains Python code (functions,
classes, variables) which you can use in your programs.\\
Instead of writing everything yourself, you can \textbf{import} modules
to get additional functionality.

\subsection{Importing a Module}\label{importing-a-module}

We use the \textbf{\texttt{import} keyword} to bring a module into our
program.\\
You can:\\
- Import the whole module → \texttt{import\ os}\\
- Import with an alias → \texttt{import\ numpy\ as\ np}\\
- Import specific functions → \texttt{from\ math\ import\ sqrt}

\subsection{\texorpdfstring{Example: The \texttt{os} Module (Operating
System)}{Example: The os Module (Operating System)}}\label{example-the-os-module-operating-system}

The \texttt{os} module provides functions to interact with the operating
system:\\
- Work with files and directories\\
- Join paths in a cross-platform way\\
- Check if files exist

\begin{Shaded}
\begin{Highlighting}[]
\ImportTok{import}\NormalTok{ os}

\NormalTok{path }\OperatorTok{=} \StringTok{"D:/GIS\_Data/OSM/shape/roads.shp"}

\BuiltInTok{print}\NormalTok{(}\StringTok{"File name:"}\NormalTok{, os.path.basename(path))}
\BuiltInTok{print}\NormalTok{(}\StringTok{"Directory:"}\NormalTok{, os.path.dirname(path))}
\BuiltInTok{print}\NormalTok{(}\StringTok{"Does file exist?"}\NormalTok{, os.path.exists(path))}
\end{Highlighting}
\end{Shaded}

\begin{verbatim}
File name: roads.shp
Directory: D:/GIS_Data/OSM/shape
Does file exist? True
\end{verbatim}

\begin{center}\rule{0.5\linewidth}{0.5pt}\end{center}

\section{Importing ArcPy in Scripts}\label{importing-arcpy-in-scripts}

ArcPy is a powerful library for GIS tasks in ArcGIS.\\
You can import it into your script like any other module.

\begin{Shaded}
\begin{Highlighting}[]
\ImportTok{import}\NormalTok{ arcpy}

\CommentTok{\# Check ArcPy version}
\NormalTok{info }\OperatorTok{=}\NormalTok{ arcpy.GetInstallInfo()}
\BuiltInTok{print}\NormalTok{(}\StringTok{"ArcPy Installed with ArcGIS Version:"}\NormalTok{, info[}\StringTok{"Version"}\NormalTok{])}
\end{Highlighting}
\end{Shaded}

\begin{verbatim}
ArcPy Installed with ArcGIS Version: 3.4
\end{verbatim}

\begin{center}\rule{0.5\linewidth}{0.5pt}\end{center}

\section{Example: A Simple Buffer
Function}\label{example-a-simple-buffer-function}

Let's create a function that performs a \textbf{buffer operation} on a
shapefile.

\begin{Shaded}
\begin{Highlighting}[]
\ImportTok{import}\NormalTok{ arcpy}

\KeywordTok{def}\NormalTok{ buffer\_features(input\_fc, output\_fc, distance):}
    \CommentTok{"""Creates a buffer around input features.}
\CommentTok{    }
\CommentTok{    Parameters:}
\CommentTok{        input\_fc (str): Input feature class (shapefile/FC)}
\CommentTok{        output\_fc (str): Output feature class path}
\CommentTok{        distance (str): Buffer distance (e.g., "50 Meters")}
\CommentTok{    """}
\NormalTok{    arcpy.analysis.Buffer(input\_fc, output\_fc, distance)}
    \BuiltInTok{print}\NormalTok{(}\SpecialStringTok{f"Buffer created: }\SpecialCharTok{\{}\NormalTok{output\_fc}\SpecialCharTok{\}}\SpecialStringTok{"}\NormalTok{)}

\CommentTok{\# Example usage (adjust file paths as needed)}
\NormalTok{arcpy.env.workspace }\OperatorTok{=} \VerbatimStringTok{r"D:/GIS\_Data/OSM/shape"}
\NormalTok{buffer\_features(}\StringTok{"roads\_utm44n.shp"}\NormalTok{, }\StringTok{"roads\_buffer\_1.shp"}\NormalTok{, }\StringTok{"50 Meters"}\NormalTok{)}
\end{Highlighting}
\end{Shaded}

\begin{Highlighting}
\textcolor{black}{ExecuteError: Failed to execute. Parameters are not valid.}
\textcolor{black}{ERROR 000725: Output Feature Class: Dataset D:/GIS\_Data/OSM/shape/roads\_buffer\_1.shp already exists.}
\textcolor{black}{Failed to execute (Buffer).}
\textcolor{black}{}\textcolor{QuartoInternalColor1}{---------------------------------------------------------------------------}\textcolor{QuartoInternalColor2}{}
\textcolor{QuartoInternalColor2}{}\textcolor{QuartoInternalColor1}{ExecuteError}\textcolor{QuartoInternalColor2}{                              Traceback (most recent call last)}
\textcolor{QuartoInternalColor2}{Cell }\textcolor{QuartoInternalColor3}{In[7], line 16}\textcolor{QuartoInternalColor2}{}
\textcolor{QuartoInternalColor2}{}\textcolor{QuartoInternalColor4}{     14}\textcolor{QuartoInternalColor2}{ }\textcolor{QuartoInternalColor5}{# Example usage (adjust file paths as needed)}\textcolor{QuartoInternalColor2}{}
\textcolor{QuartoInternalColor2}{}\textcolor{QuartoInternalColor4}{     15}\textcolor{QuartoInternalColor2}{ arcpy}\textcolor{QuartoInternalColor6}{.}\textcolor{QuartoInternalColor2}{env}\textcolor{QuartoInternalColor6}{.}\textcolor{QuartoInternalColor2}{workspace }\textcolor{QuartoInternalColor6}{=}\textcolor{QuartoInternalColor2}{ }\textcolor{QuartoInternalColor7}{r}\textcolor{QuartoInternalColor2}{}\textcolor{QuartoInternalColor7}{"}\textcolor{QuartoInternalColor2}{}\textcolor{QuartoInternalColor7}{D:/GIS\_Data/OSM/shape}\textcolor{QuartoInternalColor2}{}\textcolor{QuartoInternalColor7}{"}\textcolor{QuartoInternalColor2}{}
\textcolor{QuartoInternalColor2}{}\textcolor{QuartoInternalColor3}{---> 16}\textcolor{QuartoInternalColor2}{ buffer\_features(}\textcolor{QuartoInternalColor7}{"}\textcolor{QuartoInternalColor2}{}\textcolor{QuartoInternalColor7}{roads\_utm44n.shp}\textcolor{QuartoInternalColor2}{}\textcolor{QuartoInternalColor7}{"}\textcolor{QuartoInternalColor2}{, }\textcolor{QuartoInternalColor7}{"}\textcolor{QuartoInternalColor2}{}\textcolor{QuartoInternalColor7}{roads\_buffer\_1.shp}\textcolor{QuartoInternalColor2}{}\textcolor{QuartoInternalColor7}{"}\textcolor{QuartoInternalColor2}{, }\textcolor{QuartoInternalColor7}{"}\textcolor{QuartoInternalColor2}{}\textcolor{QuartoInternalColor7}{50 Meters}\textcolor{QuartoInternalColor2}{}\textcolor{QuartoInternalColor7}{"}\textcolor{QuartoInternalColor2}{)}
\textcolor{QuartoInternalColor2}{Cell }\textcolor{QuartoInternalColor3}{In[7], line 11}\textcolor{QuartoInternalColor2}{, in }\textcolor{QuartoInternalColor8}{buffer\_features}\textcolor{QuartoInternalColor9}{(input\_fc, output\_fc, distance)}\textcolor{QuartoInternalColor2}{}
\textcolor{QuartoInternalColor2}{}\textcolor{QuartoInternalColor4}{      3}\textcolor{QuartoInternalColor2}{ }\textcolor{QuartoInternalColor10}{def}\textcolor{QuartoInternalColor2}{ }\textcolor{QuartoInternalColor11}{buffer\_features}\textcolor{QuartoInternalColor2}{(input\_fc, output\_fc, distance):}
\textcolor{QuartoInternalColor2}{}\textcolor{QuartoInternalColor4}{      4}\textcolor{QuartoInternalColor2}{ }\textcolor{QuartoInternalColor12}{    }\textcolor{QuartoInternalColor2}{}\textcolor{QuartoInternalColor7}{"""Creates a buffer around input features.}\textcolor{QuartoInternalColor2}{}
\textcolor{QuartoInternalColor2}{}\textcolor{QuartoInternalColor4}{      5}\textcolor{QuartoInternalColor2}{ }\textcolor{QuartoInternalColor7}{    }\textcolor{QuartoInternalColor2}{}
\textcolor{QuartoInternalColor2}{}\textcolor{QuartoInternalColor4}{      6}\textcolor{QuartoInternalColor2}{ }\textcolor{QuartoInternalColor7}{    Parameters:}\textcolor{QuartoInternalColor2}{}
\textcolor{QuartoInternalColor2}{}\textcolor{QuartoInternalColor3}{   (...)}\textcolor{QuartoInternalColor2}{}
\textcolor{QuartoInternalColor2}{}\textcolor{QuartoInternalColor4}{      9}\textcolor{QuartoInternalColor2}{ }\textcolor{QuartoInternalColor7}{        distance (str): Buffer distance (e.g., "50 Meters")}\textcolor{QuartoInternalColor2}{}
\textcolor{QuartoInternalColor2}{}\textcolor{QuartoInternalColor4}{     10}\textcolor{QuartoInternalColor2}{ }\textcolor{QuartoInternalColor7}{    """}\textcolor{QuartoInternalColor2}{}
\textcolor{QuartoInternalColor2}{}\textcolor{QuartoInternalColor3}{---> 11}\textcolor{QuartoInternalColor2}{     arcpy}\textcolor{QuartoInternalColor6}{.}\textcolor{QuartoInternalColor2}{analysis}\textcolor{QuartoInternalColor6}{.}\textcolor{QuartoInternalColor2}{Buffer(input\_fc, output\_fc, distance)}
\textcolor{QuartoInternalColor2}{}\textcolor{QuartoInternalColor4}{     12}\textcolor{QuartoInternalColor2}{     }\textcolor{QuartoInternalColor10}{print}\textcolor{QuartoInternalColor2}{(}\textcolor{QuartoInternalColor7}{f}\textcolor{QuartoInternalColor2}{}\textcolor{QuartoInternalColor7}{"}\textcolor{QuartoInternalColor2}{}\textcolor{QuartoInternalColor7}{Buffer created: }\textcolor{QuartoInternalColor2}{}\textcolor{QuartoInternalColor13}{\{}\textcolor{QuartoInternalColor2}{output\_fc}\textcolor{QuartoInternalColor13}{\}}\textcolor{QuartoInternalColor2}{}\textcolor{QuartoInternalColor7}{"}\textcolor{QuartoInternalColor2}{)}
\textcolor{QuartoInternalColor2}{File }\textcolor{QuartoInternalColor3}{\textasciitilde{}\textbackslash{}AppData\textbackslash{}Local\textbackslash{}Programs\textbackslash{}ArcGIS\textbackslash{}Pro\textbackslash{}Resources\textbackslash{}ArcPy\textbackslash{}arcpy\textbackslash{}analysis.py:1699}\textcolor{QuartoInternalColor2}{, in }\textcolor{QuartoInternalColor8}{Buffer}\textcolor{QuartoInternalColor9}{(in\_features, out\_feature\_class, buffer\_distance\_or\_field, line\_side, line\_end\_type, dissolve\_option, dissolve\_field, method)}\textcolor{QuartoInternalColor2}{}
\textcolor{QuartoInternalColor2}{}\textcolor{QuartoInternalColor4}{   1697}\textcolor{QuartoInternalColor2}{     }\textcolor{QuartoInternalColor10}{return}\textcolor{QuartoInternalColor2}{ retval}
\textcolor{QuartoInternalColor2}{}\textcolor{QuartoInternalColor4}{   1698}\textcolor{QuartoInternalColor2}{ }\textcolor{QuartoInternalColor10}{except}\textcolor{QuartoInternalColor2}{ }\textcolor{QuartoInternalColor14}{Exception}\textcolor{QuartoInternalColor2}{ }\textcolor{QuartoInternalColor10}{as}\textcolor{QuartoInternalColor2}{ e:}
\textcolor{QuartoInternalColor2}{}\textcolor{QuartoInternalColor3}{-> 1699}\textcolor{QuartoInternalColor2}{     }\textcolor{QuartoInternalColor10}{raise}\textcolor{QuartoInternalColor2}{ e}
\textcolor{QuartoInternalColor2}{File }\textcolor{QuartoInternalColor3}{\textasciitilde{}\textbackslash{}AppData\textbackslash{}Local\textbackslash{}Programs\textbackslash{}ArcGIS\textbackslash{}Pro\textbackslash{}Resources\textbackslash{}ArcPy\textbackslash{}arcpy\textbackslash{}analysis.py:1681}\textcolor{QuartoInternalColor2}{, in }\textcolor{QuartoInternalColor8}{Buffer}\textcolor{QuartoInternalColor9}{(in\_features, out\_feature\_class, buffer\_distance\_or\_field, line\_side, line\_end\_type, dissolve\_option, dissolve\_field, method)}\textcolor{QuartoInternalColor2}{}
\textcolor{QuartoInternalColor2}{}\textcolor{QuartoInternalColor4}{   1677}\textcolor{QuartoInternalColor2}{ }\textcolor{QuartoInternalColor10}{from}\textcolor{QuartoInternalColor2}{ }\textcolor{QuartoInternalColor11}{arcpy}\textcolor{QuartoInternalColor2}{}\textcolor{QuartoInternalColor11}{.}\textcolor{QuartoInternalColor2}{}\textcolor{QuartoInternalColor11}{arcobjects}\textcolor{QuartoInternalColor2}{}\textcolor{QuartoInternalColor11}{.}\textcolor{QuartoInternalColor2}{}\textcolor{QuartoInternalColor11}{arcobjectconversion}\textcolor{QuartoInternalColor2}{ }\textcolor{QuartoInternalColor10}{import}\textcolor{QuartoInternalColor2}{ convertArcObjectToPythonObject}
\textcolor{QuartoInternalColor2}{}\textcolor{QuartoInternalColor4}{   1679}\textcolor{QuartoInternalColor2}{ }\textcolor{QuartoInternalColor10}{try}\textcolor{QuartoInternalColor2}{:}
\textcolor{QuartoInternalColor2}{}\textcolor{QuartoInternalColor4}{   1680}\textcolor{QuartoInternalColor2}{     retval }\textcolor{QuartoInternalColor6}{=}\textcolor{QuartoInternalColor2}{ convertArcObjectToPythonObject(}
\textcolor{QuartoInternalColor2}{}\textcolor{QuartoInternalColor3}{-> 1681}\textcolor{QuartoInternalColor2}{         gp}\textcolor{QuartoInternalColor6}{.}\textcolor{QuartoInternalColor2}{Buffer\_analysis(}
\textcolor{QuartoInternalColor2}{}\textcolor{QuartoInternalColor4}{   1682}\textcolor{QuartoInternalColor2}{             }\textcolor{QuartoInternalColor6}{*}\textcolor{QuartoInternalColor2}{gp\_fixargs(}
\textcolor{QuartoInternalColor2}{}\textcolor{QuartoInternalColor4}{   1683}\textcolor{QuartoInternalColor2}{                 (}
\textcolor{QuartoInternalColor2}{}\textcolor{QuartoInternalColor4}{   1684}\textcolor{QuartoInternalColor2}{                     in\_features,}
\textcolor{QuartoInternalColor2}{}\textcolor{QuartoInternalColor4}{   1685}\textcolor{QuartoInternalColor2}{                     out\_feature\_class,}
\textcolor{QuartoInternalColor2}{}\textcolor{QuartoInternalColor4}{   1686}\textcolor{QuartoInternalColor2}{                     buffer\_distance\_or\_field,}
\textcolor{QuartoInternalColor2}{}\textcolor{QuartoInternalColor4}{   1687}\textcolor{QuartoInternalColor2}{                     line\_side,}
\textcolor{QuartoInternalColor2}{}\textcolor{QuartoInternalColor4}{   1688}\textcolor{QuartoInternalColor2}{                     line\_end\_type,}
\textcolor{QuartoInternalColor2}{}\textcolor{QuartoInternalColor4}{   1689}\textcolor{QuartoInternalColor2}{                     dissolve\_option,}
\textcolor{QuartoInternalColor2}{}\textcolor{QuartoInternalColor4}{   1690}\textcolor{QuartoInternalColor2}{                     dissolve\_field,}
\textcolor{QuartoInternalColor2}{}\textcolor{QuartoInternalColor4}{   1691}\textcolor{QuartoInternalColor2}{                     method,}
\textcolor{QuartoInternalColor2}{}\textcolor{QuartoInternalColor4}{   1692}\textcolor{QuartoInternalColor2}{                 ),}
\textcolor{QuartoInternalColor2}{}\textcolor{QuartoInternalColor4}{   1693}\textcolor{QuartoInternalColor2}{                 }\textcolor{QuartoInternalColor10}{True}\textcolor{QuartoInternalColor2}{,}
\textcolor{QuartoInternalColor2}{}\textcolor{QuartoInternalColor4}{   1694}\textcolor{QuartoInternalColor2}{             )}
\textcolor{QuartoInternalColor2}{}\textcolor{QuartoInternalColor4}{   1695}\textcolor{QuartoInternalColor2}{         )}
\textcolor{QuartoInternalColor2}{}\textcolor{QuartoInternalColor4}{   1696}\textcolor{QuartoInternalColor2}{     )}
\textcolor{QuartoInternalColor2}{}\textcolor{QuartoInternalColor4}{   1697}\textcolor{QuartoInternalColor2}{     }\textcolor{QuartoInternalColor10}{return}\textcolor{QuartoInternalColor2}{ retval}
\textcolor{QuartoInternalColor2}{}\textcolor{QuartoInternalColor4}{   1698}\textcolor{QuartoInternalColor2}{ }\textcolor{QuartoInternalColor10}{except}\textcolor{QuartoInternalColor2}{ }\textcolor{QuartoInternalColor14}{Exception}\textcolor{QuartoInternalColor2}{ }\textcolor{QuartoInternalColor10}{as}\textcolor{QuartoInternalColor2}{ e:}
\textcolor{QuartoInternalColor2}{File }\textcolor{QuartoInternalColor3}{\textasciitilde{}\textbackslash{}AppData\textbackslash{}Local\textbackslash{}Programs\textbackslash{}ArcGIS\textbackslash{}Pro\textbackslash{}Resources\textbackslash{}ArcPy\textbackslash{}arcpy\textbackslash{}geoprocessing\textbackslash{}\_base.py:532}\textcolor{QuartoInternalColor2}{, in }\textcolor{QuartoInternalColor8}{Geoprocessor.\_\_getattr\_\_.<locals>.<lambda>}\textcolor{QuartoInternalColor9}{(*args)}\textcolor{QuartoInternalColor2}{}
\textcolor{QuartoInternalColor2}{}\textcolor{QuartoInternalColor4}{    530}\textcolor{QuartoInternalColor2}{ val }\textcolor{QuartoInternalColor6}{=}\textcolor{QuartoInternalColor2}{ }\textcolor{QuartoInternalColor10}{getattr}\textcolor{QuartoInternalColor2}{(}\textcolor{QuartoInternalColor10}{self}\textcolor{QuartoInternalColor2}{}\textcolor{QuartoInternalColor6}{.}\textcolor{QuartoInternalColor2}{\_gp, attr)}
\textcolor{QuartoInternalColor2}{}\textcolor{QuartoInternalColor4}{    531}\textcolor{QuartoInternalColor2}{ }\textcolor{QuartoInternalColor10}{if}\textcolor{QuartoInternalColor2}{ }\textcolor{QuartoInternalColor10}{callable}\textcolor{QuartoInternalColor2}{(val):}
\textcolor{QuartoInternalColor2}{}\textcolor{QuartoInternalColor3}{--> 532}\textcolor{QuartoInternalColor2}{     }\textcolor{QuartoInternalColor10}{return}\textcolor{QuartoInternalColor2}{ }\textcolor{QuartoInternalColor10}{lambda}\textcolor{QuartoInternalColor2}{ }\textcolor{QuartoInternalColor6}{*}\textcolor{QuartoInternalColor2}{args: val(}\textcolor{QuartoInternalColor6}{*}\textcolor{QuartoInternalColor2}{gp\_fixargs(args, }\textcolor{QuartoInternalColor10}{True}\textcolor{QuartoInternalColor2}{))}
\textcolor{QuartoInternalColor2}{}\textcolor{QuartoInternalColor4}{    533}\textcolor{QuartoInternalColor2}{ }\textcolor{QuartoInternalColor10}{else}\textcolor{QuartoInternalColor2}{:}
\textcolor{QuartoInternalColor2}{}\textcolor{QuartoInternalColor4}{    534}\textcolor{QuartoInternalColor2}{     }\textcolor{QuartoInternalColor10}{return}\textcolor{QuartoInternalColor2}{ convertArcObjectToPythonObject(val)}
\textcolor{QuartoInternalColor2}{}\textcolor{QuartoInternalColor1}{ExecuteError}\textcolor{QuartoInternalColor2}{: Failed to execute. Parameters are not valid.}
\textcolor{QuartoInternalColor2}{ERROR 000725: Output Feature Class: Dataset D:/GIS\_Data/OSM/shape/roads\_buffer\_1.shp already exists.}
\textcolor{QuartoInternalColor2}{Failed to execute (Buffer).}
\end{Highlighting}

\begin{center}\rule{0.5\linewidth}{0.5pt}\end{center}

\section{Summary}\label{summary-3}

In this chapter, we learned:\\
- How to define and use \textbf{functions}.\\
- How to use built-in modules like \textbf{os} and \textbf{sys}.\\
- How to import and use \textbf{ArcPy} in scripts.\\
- How to write a \textbf{buffer function} for automation.

\begin{center}\rule{0.5\linewidth}{0.5pt}\end{center}

\section{Exercise}\label{exercise-3}

Try the following tasks:

\begin{enumerate}
\def\labelenumi{\arabic{enumi}.}
\tightlist
\item
  Write a function that calculates the area of a rectangle (length ×
  width).\\
\item
  Use the \texttt{os} module to list all files in a folder of your
  choice.\\
\item
  Modify the buffer function to accept multiple shapefiles and create
  buffers for each one.
\end{enumerate}

\begin{center}\rule{0.5\linewidth}{0.5pt}\end{center}

\chapter{Error Handling and
Exceptions}\label{error-handling-and-exceptions}

When writing Python scripts, errors are inevitable. For example, a file
might be missing, a dataset may not load correctly, or a user may type
the wrong parameter.\\
Python provides a mechanism called \textbf{exceptions} to handle such
errors gracefully instead of letting the program crash.

In GIS scripting with ArcPy, error handling is especially important
because geoprocessing tools often fail if inputs are invalid or if files
already exist.

\begin{center}\rule{0.5\linewidth}{0.5pt}\end{center}

\section{What is an Exception?}\label{what-is-an-exception}

An \textbf{exception} is an error that occurs during program
execution.\\
Common Python exceptions include:

\begin{itemize}
\tightlist
\item
  \texttt{FileNotFoundError} → File does not exist\\
\item
  \texttt{ZeroDivisionError} → Division by zero\\
\item
  \texttt{ValueError} → Wrong value or data type\\
\item
  \texttt{TypeError} → Wrong data type in operation\\
\item
  \texttt{arcpy.ExecuteError} → ArcPy tool execution failed
\end{itemize}

\begin{center}\rule{0.5\linewidth}{0.5pt}\end{center}

\section{The try--except Block}\label{the-tryexcept-block}

We can use a \textbf{try--except} block to catch and handle exceptions.

\begin{Shaded}
\begin{Highlighting}[]
\ControlFlowTok{try}\NormalTok{:}
\NormalTok{    x }\OperatorTok{=} \DecValTok{10} \OperatorTok{/} \DecValTok{0}   \CommentTok{\# This will cause ZeroDivisionError}
\ControlFlowTok{except} \PreprocessorTok{ZeroDivisionError}\NormalTok{:}
    \BuiltInTok{print}\NormalTok{(}\StringTok{"You cannot divide by zero!"}\NormalTok{)}
\end{Highlighting}
\end{Shaded}

\begin{verbatim}
You cannot divide by zero!
\end{verbatim}

\begin{center}\rule{0.5\linewidth}{0.5pt}\end{center}

\section{Handling Multiple
Exceptions}\label{handling-multiple-exceptions}

We can catch different exceptions with multiple \texttt{except} blocks.

\begin{Shaded}
\begin{Highlighting}[]
\ControlFlowTok{try}\NormalTok{:}
\NormalTok{    number }\OperatorTok{=} \BuiltInTok{int}\NormalTok{(}\StringTok{"abc"}\NormalTok{)   }\CommentTok{\# This will cause ValueError}
\ControlFlowTok{except} \PreprocessorTok{ValueError}\NormalTok{:}
    \BuiltInTok{print}\NormalTok{(}\StringTok{"Invalid number!"}\NormalTok{)}
\ControlFlowTok{except} \PreprocessorTok{TypeError}\NormalTok{:}
    \BuiltInTok{print}\NormalTok{(}\StringTok{"Type error occurred!"}\NormalTok{)}
\end{Highlighting}
\end{Shaded}

\begin{verbatim}
Invalid number!
\end{verbatim}

\begin{center}\rule{0.5\linewidth}{0.5pt}\end{center}

\section{Using finally}\label{using-finally}

The \texttt{finally} block always runs, whether an error occurred or
not.\\
It is useful for cleanup (closing files, releasing resources, etc.).

\begin{Shaded}
\begin{Highlighting}[]
\ControlFlowTok{try}\NormalTok{:}
    \BuiltInTok{file} \OperatorTok{=} \BuiltInTok{open}\NormalTok{(}\StringTok{"data.txt"}\NormalTok{)}
\NormalTok{    content }\OperatorTok{=} \BuiltInTok{file}\NormalTok{.read()}
\ControlFlowTok{except} \PreprocessorTok{FileNotFoundError}\NormalTok{:}
    \BuiltInTok{print}\NormalTok{(}\StringTok{"File not found!"}\NormalTok{)}
\ControlFlowTok{finally}\NormalTok{:}
    \BuiltInTok{print}\NormalTok{(}\StringTok{"This block always executes."}\NormalTok{)}
\end{Highlighting}
\end{Shaded}

\begin{verbatim}
File not found!
This block always executes.
\end{verbatim}

\begin{center}\rule{0.5\linewidth}{0.5pt}\end{center}

\section{Raising Exceptions}\label{raising-exceptions}

Sometimes we want to raise an exception intentionally if something is
wrong.

\begin{Shaded}
\begin{Highlighting}[]
\KeywordTok{def}\NormalTok{ calculate\_area(length, width):}
    \ControlFlowTok{if}\NormalTok{ length }\OperatorTok{\textless{}=} \DecValTok{0} \KeywordTok{or}\NormalTok{ width }\OperatorTok{\textless{}=} \DecValTok{0}\NormalTok{:}
        \ControlFlowTok{raise} \PreprocessorTok{ValueError}\NormalTok{(}\StringTok{"Length and width must be positive numbers"}\NormalTok{)}
    \ControlFlowTok{return}\NormalTok{ length }\OperatorTok{*}\NormalTok{ width}

\CommentTok{\# Uncomment below to test}
\CommentTok{\# print(calculate\_area({-}5, 10))}
\end{Highlighting}
\end{Shaded}

\begin{center}\rule{0.5\linewidth}{0.5pt}\end{center}

\section{Error Handling in ArcPy}\label{error-handling-in-arcpy}

ArcPy provides a special exception called
\textbf{\texttt{arcpy.ExecuteError}}, which is raised when a
geoprocessing tool fails.

\begin{Shaded}
\begin{Highlighting}[]
\ImportTok{import}\NormalTok{ arcpy}

\NormalTok{arcpy.env.workspace }\OperatorTok{=} \VerbatimStringTok{r"D:/GIS\_Data/OSM/shape}
\ErrorTok{try:}
    \CommentTok{\# Try running a buffer on a non{-}existent dataset}
\NormalTok{    arcpy.Buffer\_analysis(}\StringTok{"non\_existent.shp"}\NormalTok{, }\StringTok{"output.shp"}\NormalTok{, }\StringTok{"50 Meters"}\NormalTok{)}
\ControlFlowTok{except}\NormalTok{ arcpy.ExecuteError:}
    \BuiltInTok{print}\NormalTok{(}\StringTok{"ArcPy tool failed:"}\NormalTok{, arcpy.GetMessages(}\DecValTok{2}\NormalTok{))  }\CommentTok{\# Get error messages}
\ControlFlowTok{except} \PreprocessorTok{Exception} \ImportTok{as}\NormalTok{ e:}
    \BuiltInTok{print}\NormalTok{(}\StringTok{"Unexpected error:"}\NormalTok{, e)}
\end{Highlighting}
\end{Shaded}

\begin{Highlighting}
\textcolor{black}{SyntaxError: unterminated string literal (detected at line 3) (2187730895.py, line 3)}
\textcolor{black}{}\textcolor{QuartoInternalColor15}{  Cell }\textcolor{QuartoInternalColor3}{In[5], line 3}\textcolor{QuartoInternalColor15}{}\textcolor{QuartoInternalColor2}{}
\textcolor{QuartoInternalColor2}{}\textcolor{QuartoInternalColor16}{    arcpy.env.workspace = r"D:/GIS\_Data/OSM/shape}\textcolor{QuartoInternalColor2}{}
\textcolor{QuartoInternalColor2}{}\textcolor{QuartoInternalColor17}{                          \textasciicaret{}}\textcolor{QuartoInternalColor2}{}
\textcolor{QuartoInternalColor2}{}\textcolor{QuartoInternalColor1}{SyntaxError}\textcolor{QuartoInternalColor2}{}\textcolor{QuartoInternalColor1}{:}\textcolor{QuartoInternalColor2}{ unterminated string literal (detected at line 3)}
\end{Highlighting}

\begin{center}\rule{0.5\linewidth}{0.5pt}\end{center}

\section{Practical Example: Safe Buffer
Script}\label{practical-example-safe-buffer-script}

\begin{Shaded}
\begin{Highlighting}[]
\ImportTok{import}\NormalTok{ arcpy}

\KeywordTok{def}\NormalTok{ safe\_buffer(input\_fc, output\_fc, distance):}
    \ControlFlowTok{try}\NormalTok{:}
\NormalTok{        arcpy.Buffer\_analysis(input\_fc, output\_fc, distance)}
        \BuiltInTok{print}\NormalTok{(}\SpecialStringTok{f"Buffer created: }\SpecialCharTok{\{}\NormalTok{output\_fc}\SpecialCharTok{\}}\SpecialStringTok{"}\NormalTok{)}
    \ControlFlowTok{except}\NormalTok{ arcpy.ExecuteError:}
        \BuiltInTok{print}\NormalTok{(}\StringTok{"Buffer tool failed:"}\NormalTok{, arcpy.GetMessages(}\DecValTok{2}\NormalTok{))}
    \ControlFlowTok{except} \PreprocessorTok{Exception} \ImportTok{as}\NormalTok{ e:}
        \BuiltInTok{print}\NormalTok{(}\StringTok{"Unexpected error:"}\NormalTok{, e)}

\CommentTok{\# Example usage}
\NormalTok{safe\_buffer(}\StringTok{"roads.shp"}\NormalTok{, }\StringTok{"roads\_buffer.shp"}\NormalTok{, }\StringTok{"50 Meters"}\NormalTok{)}
\end{Highlighting}
\end{Shaded}

\begin{verbatim}
Buffer tool failed: Failed to execute. Parameters are not valid.
ERROR 000732: Input Features: Dataset roads.shp does not exist or is not supported
Failed to execute (Buffer).
\end{verbatim}

\begin{center}\rule{0.5\linewidth}{0.5pt}\end{center}

\section{Common Exceptions Reference
Table}\label{common-exceptions-reference-table}

\begin{longtable}[]{@{}
  >{\raggedright\arraybackslash}p{(\linewidth - 4\tabcolsep) * \real{0.4182}}
  >{\raggedright\arraybackslash}p{(\linewidth - 4\tabcolsep) * \real{0.2364}}
  >{\raggedright\arraybackslash}p{(\linewidth - 4\tabcolsep) * \real{0.3455}}@{}}
\toprule\noalign{}
\begin{minipage}[b]{\linewidth}\raggedright
Exception Type
\end{minipage} & \begin{minipage}[b]{\linewidth}\raggedright
Description
\end{minipage} & \begin{minipage}[b]{\linewidth}\raggedright
GIS/ArcPy Example
\end{minipage} \\
\midrule\noalign{}
\endhead
\bottomrule\noalign{}
\endlastfoot
\texttt{FileNotFoundError} & File or directory not found & Trying to
open a missing shapefile \\
\texttt{PermissionError} & No permission to read/write & Attempting to
overwrite a locked file \\
\texttt{ZeroDivisionError} & Division by zero & Calculating density with
area = 0 \\
\texttt{ValueError} & Wrong value or type & Passing a string instead of
a number \\
\texttt{TypeError} & Operation on wrong type & Adding a string and an
integer \\
\texttt{IndexError} & Index out of range & Accessing a list element that
does not exist \\
\texttt{KeyError} & Key not found in dictionary & Accessing a missing
field in a dict \\
\texttt{MemoryError} & Out of memory & Processing very large raster in
memory \\
\texttt{arcpy.ExecuteError} & ArcPy tool execution failed & Buffer or
Clip failed due to bad input \\
\texttt{arcpy.ExecuteWarning} & Tool ran with warnings & Projection
mismatch, empty outputs \\
\end{longtable}

💡 Keep this table handy when debugging scripts. Most issues in Python
or ArcPy will map to one of these exceptions.

\begin{center}\rule{0.5\linewidth}{0.5pt}\end{center}

\section{Summary}\label{summary-4}

In this chapter, we learned:\\
- What exceptions are and why they occur.\\
- How to use \textbf{try--except--finally} for error handling.\\
- How to raise exceptions deliberately.\\
- How to handle \textbf{ArcPy errors} with
\texttt{arcpy.ExecuteError}.\\
- A practical GIS example: a safe buffer script with error handling.\\
- A quick reference table of \textbf{common exceptions} in Python and
ArcPy.

\begin{center}\rule{0.5\linewidth}{0.5pt}\end{center}

\section{Exercises}\label{exercises}

\begin{enumerate}
\def\labelenumi{\arabic{enumi}.}
\tightlist
\item
  Write a function that divides two numbers and uses \texttt{try–except}
  to avoid division by zero.\\
\item
  Modify the safe buffer script to check if the output file already
  exists. If it does, print a warning instead of overwriting.\\
\item
  Use \texttt{arcpy.Clip\_analysis()} in a try--except block and handle
  both ArcPy errors and general exceptions.
\end{enumerate}

\begin{center}\rule{0.5\linewidth}{0.5pt}\end{center}


\backmatter


\end{document}
